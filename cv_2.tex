%%%%%%%%%%%%%%%%%%%%%%%%%%%%%%%%%%%%%%%%%%%%%%%%%%%%%%%%%%%%%%%%%%%%%%%%%%%%%%%%
% Medium Length Graduate Curriculum Vitae
% LaTeX Template
% Version 1.2 (3/28/15)
%
% This template has been downloaded from:
% http://www.LaTeXTemplates.com
%
% Original author:
% Rensselaer Polytechnic Institute 
% (http://www.rpi.edu/dept/arc/training/latex/resumes/)
%
% Modified by:
% Daniel L Marks <xleafr@gmail.com> 3/28/2015
%
% Important note:
% This template requires the res.cls file to be in the same directory as the
% .tex file. The res.cls file provides the resume style used for structuring the
% document.
%
%%%%%%%%%%%%%%%%%%%%%%%%%%%%%%%%%%%%%%%%%%%%%%%%%%%%%%%%%%%%%%%%%%%%%%%%%%%%%%%%

%-------------------------------------------------------------------------------
%	PACKAGES AND OTHER DOCUMENT CONFIGURATIONS
%-------------------------------------------------------------------------------

%%%%%%%%%%%%%%%%%%%%%%%%%%%%%%%%%%%%%%%%%%%%%%%%%%%%%%%%%%%%%%%%%%%%%%%%%%%%%%%%
% You can have multiple style options the legal options ones are:
%
%   centered:	the name and address are centered at the top of the page 
%				(default)
%
%   line:		the name is the left with a horizontal line then the address to
%				the right
%
%   overlapped:	the section titles overlap the body text (default)
%
%   margin:		the section titles are to the left of the body text
%		
%   11pt:		use 11 point fonts instead of 10 point fonts
%
%   12pt:		use 12 point fonts instead of 10 point fonts
%
%%%%%%%%%%%%%%%%%%%%%%%%%%%%%%%%%%%%%%%%%%%%%%%%%%%%%%%%%%%%%%%%%%%%%%%%%%%%%%%%

\documentclass[margin]{res}  

% Default font is the helvetica postscript font
\usepackage{helvet}
\usepackage{hyperref}
\hypersetup{
    colorlinks=true,    % Keep the box around the links
    pdfborder={0 0 1},   % Border thickness
    urlcolor=blue,       % Color for URL text
    linkcolor=blue,      % Color for link text
    filecolor=magenta,   % Color for file links
    pdftitle={My Resume}, % PDF metadata
    pdfauthor={Ramon da Silva Torres}  % PDF author metadata
}
\usepackage[normalem]{ulem}  % Load ulem package (normalem prevents affecting \emph)
\newcommand{\ulhref}[2]{\href{#1}{\uline{#2}}} % Define a custom command


% Increase text height
\textheight=700pt

\begin{document}

%-------------------------------------------------------------------------------
%	NAME AND ADDRESS SECTION
%-------------------------------------------------------------------------------
\name{Ramon da Silva Torres}

\address{
Contato: (31) 98806-9741 \\ 
rsnatorres@gmail.com
\\ Idade: 32 anos 
\\ Residência: Rio de Janeiro/RJ
}
\address{
\href{https://github.com/rsnatorres}{GitHub}\\ \href{https://www.linkedin.com/in/ramon-torres-5b1164119/}{Linkedin}\\
\href{http://lattes.cnpq.br/4650341120288709}{Lattes}
}

%-------------------------------------------------------------------------------

\begin{resume}

%-------------------------------------------------------------------------------
%	EDUCATION SECTION
%-------------------------------------------------------------------------------
\section{FORMAÇÃO}
\textbf{Universidade Federal de Minas Gerais (UFMG)}\\
{\sl Doutorado em Ciências Econômicas}, 2021 - atualmente (qualificado)
\\
{\sl Mestrado em Ciências Econômicas}, 2019 - 2021
\\
{\sl Bacharel em Ciências Econômicas}, 2011 - 2015
\\
\textbf{Ibmec}\\
{\sl MBA em Gestão de Projetos}, 2019 - 2020
\\


%-------------------------------------------------------------------------------

%-------------------------------------------------------------------------------
%	EXPERIENCE SECTION
%-------------------------------------------------------------------------------
\begin{format}
\title{l}\employer{r}\\
\dates{l}\location{r}\\
\body\\
\end{format}

\section{EXPERIÊNCIA}

\employer{DataViva}
\location{Remoto}
\dates{Nov/2022 - atualmente}
\title{\textbf{Gestor de Dados e Líder Técnico}}
\begin{position}
Gestor do time de desenvolvimento (seis desenvolvedores e dois pesquisadores) da plataforma de visualização de dados \href{https://www.dataviva.info/pt/}{DataViva} e líder técnico dos projetos de ciência de dados. Entre os papeis de gestor destacam-se: i) a condução dos eventos de metodologia ágil do \textit{framework scrum} (\textit{dailies}, \textit{plannings}, \textit{sprint reviews}) junto a equipe e clientes; ii) a elaboração dos Planos de Desenvolvimento Individuais (PDI); iii) a elaboração de projetos para disputa de licitação; iv) a gestão de contratos (prazos e custos). Entre os papeis de líder técnico destacam-se: i) construção de algoritmos em \textit{Python} para tratamento de dados; ii) construção de modelos de \textit{machine learning}; iii) decisões e implementação em código da arquitetura do \textit{pipeline} de ingestão de dados.
\end{position}

\employer{Fundação Vanzolini}
\location{Remoto}
\dates{Set/2023 - Mar/2025}
\title{\textbf{Orientador de MBA}}
\begin{position}
Orientação de Trabalhos de Conclusão de Curso dos MBA's de Data Science e Analytics para Operações e Gestão de Projetos da \href{https://poliusppro.com/}{POLI USP PRO}. 

\end{position}


\employer{Tavus LTDA}
\location{Remoto}
\dates{Jul/2021 - atualmente}
\title{\textbf{Consultor de Projetos de Ciência de Dados}}
\begin{position}
Prestação de consultoria em serviços diversos na área de ciência de dados tais como: (i) captação de dados via \textit{webcrawler}; (ii) construção e manutenção de banco de dados; (iii) limpeza, transformação e análise de dados; (iv) construção de visualizações, \textit{dashboards} e relatórios; (v) construção de modelos de \textit{machine learning} (e inteligência artificial) baseados em análise de regressão e agrupamento. 
Todos os trabalhos tem sido desenvolvidos através das linguagens \textit{Python}, SQL e R.
\end{position}


\employer{Fundação João Pinheiro}
\location{Belo Horizonte/MG}
\dates{Mar/2021 - Out/2021}
\title{\textbf{Pesquisador}}
\begin{position}
Análise dos microdados do CENSO/IBGE (2000 e 2010) e de bases de dados do FGTS; Construção de Projeções Econômico-Financeiras; Construção de algoritmos em \textit{Python} para análise estatística; Elaboração de estudos setoriais. 
\end{position}

\employer{SEBRAE}
\location{Belo Horizonte/MG}
\dates{Set/2019 - Fev/2021}
\title{\textbf{Consultor em Gestão da Inovação}}
\begin{position}
Atuação em consultorias em gestão da inovação para negócios  no Programa Agente Local de Inovação (ALI); Elaboração de estudo sobre a economia com base em dados primários e secundários (Censo Agropecuário/IBGE, Cempre/IBGE, RAIS, e Inep/MEC); Construção de Indicadores (KPI’s) para monitoramento dos resultados do negócio; Planejamento de pesquisas de mercado.
\end{position}

\employer{CEDEPLAR}
\location{Belo Horizonte/MG}
\dates{Jan/2017 - Set/2018}
\title{\textbf{Pesquisador}}
\begin{position}
Orientação/Supervisão de dois bolsistas de graduação em pesquisa de campo; Levantamento, sistematização e tratamento de dados quantitativos e qualitativos; Análise de microdados (RAIS, ITBI-BH, CNAE, CENSO/IBGE) através de softwares estatísticos (Stata, SPSS) e linguagem de programação R; Construção e gestão de Banco de Dados (Access); Construção de visualizações e produção de relatórios.
\end{position}

\employer{FAPEMIG}
\location{Belo Horizonte/MG}
\dates{Fev/2017 - Mai/2018}
\title{\textbf{Pesquisador}}
\begin{position}
Aplicação de algoritmos de ciência de dados (\textit{k-means}, \textit{random forests}, regressão linear) para fins de testes de conceito da Plataforma Lemonade.
\end{position}


%-------------------------------------------------------------------------------
%	TEACHING EXPERIENCE
%-------------------------------------------------------------------------------

\section{EXPERIÊNCIA DOCENTE}

\textbf{Disciplina  Python para Ciência de Dados} | CEDEPLAR  (2023/01, 2023/02, 2024/01) | 30h \\
\textbf{Minicurso Python para Ciência de Dados} | CEDEPLAR (2022/02) | UFSJ (2023/02)  | 12h \\
\textbf{Curso Dinâmica Imobiliária} | IPEAD (2022/01) | 30h \\

%-------------------------------------------------------------------------------
%	SKILLS SECTION
%-------------------------------------------------------------------------------
\section{HABILIDADES}

\textbf{Linguagens de Programação}: Python, JavaScript, R, SQL, Bash, \LaTeX.
\\
\textbf{Tratamento de Dados}: Pandas, Geopandas, NumPy, Spacy (NLP), Spark.
\\
\textbf{Análise Estatística}: Statsmodels (Econometria), Scikit-learn, Scipy.
\\
\textbf{Visualização de Dados}: Matplotlib, Plotly, Seaborn, Ggplot.
\\
\textbf{Desenvolvimento Web}: HTML, CSS, React, Tailwind, Node, Express.
\\
\textbf{Ferramentas}: Git, Visual Studio, Jupyter Notebooks, R-Studio, MySQL, DBeaver, MongoDB, Postman.
\\
\textbf{Cloud}: AWS, Heroku.
\\
\textbf{Sistemas Operacionais}: Windows, Linux.
\\
\textbf{Econometria}: Modelos de Regressão Múltipla; Modelos de Regressão com variável dependente binária (LOGIT, PROBIT, ordinais, multinomiais); Séries Temporais (ARIMA, SARIMAX, VAR, \textit{Local Projections}).
\\
\textbf{Machine Learning}: Aplicação de modelos não supervisionados (k-\textit{means}, \textit{Negative Matrix Factorization}, \textit{Latent Dirichlet Allocation}); Aplicação de modelos supervisionados (\textit{XGBoost}, \textit{Random Forest}, Redes Neurais com TensorFlow/PyTorch).
\\
\textbf{Inteligência Artificial (IA)/\textit{Deep Learning}}: Aplicação e integração de modelos pré-treinados de IA/\textit{Deep Learning} \textit{open source} disponibilizados na plataforma \textit{Hugging Face}. Experiência com modelos de geração de texto e tradução (GPT, BART, T5); Modelos de \textit{embeddings} de palavras e sentenças (Word2Vec, FastText, BERT, Sentence-BERT);

%-------------------------------------------------------------------------------
%	PORTFÓLIO
%-------------------------------------------------------------------------------
\section{PORTFÓLIO PÚBLICO}

\href{https://github.com/rsnatorres/phd_thesis}{\textbf{Mineração de \textit{tweets} para extração de índices de sentimento}}: Extração de 55 milhões de \textit{tweets} contendo palavras-chave que versem sobre temas econômicos. Emprego de técnicas de Processamento de Linguagem Natural (PLN) para extração de polaridade e agregação em índices de sentimento/incerteza econômicos. Utilização dos índices como insumo em modelos de previsão (VAR e \textit{Local Projections}).

\href{https://github.com/rsnatorres/curso_python_dados}{\textbf{Curso de \textit{Python} para Ciência de Dados}}: Material utilizado nas aulas do curso. O curso é ofertado através da ferramenta Colab instanciada no próprio Drive dos alunos (requer uma conta @gmail.com). O curso cobre os seguintes tópicos: 1) Introdução; 2) \textit{Dataframes}; 3) Filtros e tabelas; 4) Visualização; 5) Transformações de dados; 6) \textit{Merges}; 7) Mapas; 8) Econometria; 9) Automatização de tarefas.


%-------------------------------------------------------------------------------
%	PUBLICATIONS SECTION
%-------------------------------------------------------------------------------
\section{PUBLICAÇÕES}

\textbf{TORRES, Ramon; TONUCCI, João; ALMEIDA, Renan.} \textit{Financeirização do imobiliário no Brasil: uma análise dos Certificados de Recebíveis Imobiliários (2005-2020)}. Cadernos Metrópole, 2022. \href{https://www.scielo.br/j/cm/a/Vb9yvGf34vP9sthm5RXhbBy/?lang=pt}{Scielo}.


\textbf{ALMEIDA, Renan P.; Patrício, Pedro A.; BRANDAO, M. B.; TORRES, Ramon.} \textit{Can economic development policy trigger gentrification?} Economy and Space, 2021. \href{https://journals.sagepub.com/doi/10.1177/0308518X211050076}{Sage Journals}.

\textbf{ALMEIDA, Renan; BRANDÃO, Marcelo; TORRES, Ramon; PATRÍCIO, Pedro; AMARAL, Pedro.} \textit{An assessment of the impacts of large scale urban projects on land values}. Papers in Regional Science, 2020. \href{https://rsaiconnect.onlinelibrary.wiley.com/doi/10.1111/pirs.12572}{Wiley}.


%-------------------------------------------------------------------------------
%	LANGUAGES SECTION
%-------------------------------------------------------------------------------
\section{LINGUAS}
\textbf{Inglês}: Avançado\\
\textbf{Espanhol}: Básico

%-------------------------------------------------------------------------------
\end{resume}
\end{document}
